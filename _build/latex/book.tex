%% Generated by Sphinx.
\def\sphinxdocclass{jupyterBook}
\documentclass[letterpaper,10pt,english]{jupyterBook}
\ifdefined\pdfpxdimen
   \let\sphinxpxdimen\pdfpxdimen\else\newdimen\sphinxpxdimen
\fi \sphinxpxdimen=.75bp\relax
\ifdefined\pdfimageresolution
    \pdfimageresolution= \numexpr \dimexpr1in\relax/\sphinxpxdimen\relax
\fi
%% let collapsible pdf bookmarks panel have high depth per default
\PassOptionsToPackage{bookmarksdepth=5}{hyperref}
%% turn off hyperref patch of \index as sphinx.xdy xindy module takes care of
%% suitable \hyperpage mark-up, working around hyperref-xindy incompatibility
\PassOptionsToPackage{hyperindex=false}{hyperref}
%% memoir class requires extra handling
\makeatletter\@ifclassloaded{memoir}
{\ifdefined\memhyperindexfalse\memhyperindexfalse\fi}{}\makeatother

\PassOptionsToPackage{warn}{textcomp}

\catcode`^^^^00a0\active\protected\def^^^^00a0{\leavevmode\nobreak\ }
\usepackage{cmap}
\usepackage{fontspec}
\defaultfontfeatures[\rmfamily,\sffamily,\ttfamily]{}
\usepackage{amsmath,amssymb,amstext}
\usepackage{polyglossia}
\setmainlanguage{english}



\setmainfont{FreeSerif}[
  Extension      = .otf,
  UprightFont    = *,
  ItalicFont     = *Italic,
  BoldFont       = *Bold,
  BoldItalicFont = *BoldItalic
]
\setsansfont{FreeSans}[
  Extension      = .otf,
  UprightFont    = *,
  ItalicFont     = *Oblique,
  BoldFont       = *Bold,
  BoldItalicFont = *BoldOblique,
]
\setmonofont{FreeMono}[
  Extension      = .otf,
  UprightFont    = *,
  ItalicFont     = *Oblique,
  BoldFont       = *Bold,
  BoldItalicFont = *BoldOblique,
]



\usepackage[Bjarne]{fncychap}
\usepackage[,numfigreset=2,mathnumfig]{sphinx}

\fvset{fontsize=\small}
\usepackage{geometry}


% Include hyperref last.
\usepackage{hyperref}
% Fix anchor placement for figures with captions.
\usepackage{hypcap}% it must be loaded after hyperref.
% Set up styles of URL: it should be placed after hyperref.
\urlstyle{same}


\usepackage{sphinxmessages}



        % Start of preamble defined in sphinx-jupyterbook-latex %
         \usepackage[Latin,Greek]{ucharclasses}
        \usepackage{unicode-math}
        % fixing title of the toc
        \addto\captionsenglish{\renewcommand{\contentsname}{Contents}}
        \hypersetup{
            pdfencoding=auto,
            psdextra
        }
        % End of preamble defined in sphinx-jupyterbook-latex %
        

\title{Introduction}
\date{Apr 16, 2023}
\release{}
\author{Florian Schnabel}
\newcommand{\sphinxlogo}{\vbox{}}
\renewcommand{\releasename}{}
\makeindex
\begin{document}

\pagestyle{empty}
\sphinxmaketitle
\pagestyle{plain}
\sphinxtableofcontents
\pagestyle{normal}
\phantomsection\label{\detokenize{intro::doc}}


\sphinxAtStartPar
This is a books serves to collect my submissions for \sphinxstyleemphasis{numerical methods in building science}

\sphinxstepscope




\section{Exercise 1 \sphinxhyphen{} Heat Conduction with multiple layers and variable properties}
\label{\detokenize{Aufgabe1:exercise-1-heat-conduction-with-multiple-layers-and-variable-properties}}\label{\detokenize{Aufgabe1::doc}}
\sphinxAtStartPar
This script aims to calculate the temperature field across a multilayered building component. The building component can have an arbitrary number of layers with varying properties. Furthermore,
the thickness of the layer may change as well as the chosen number of finite volumes used to discretize
the domain. Such a multilayered building component is depicted in \hyperref[\detokenize{Aufgabe1:fig-multilaycomp}]{Fig.\@ \ref{\detokenize{Aufgabe1:fig-multilaycomp}}}.

\begin{figure}[htbp]
\centering
\capstart

\noindent\sphinxincludegraphics[width=350\sphinxpxdimen]{{BuildingComponent}.png}
\caption{Multilayered building Component and discretisation requirement {[}excersise Description{]}}\label{\detokenize{Aufgabe1:fig-multilaycomp}}\end{figure}

\sphinxAtStartPar
To allow a numerical solution of the problem the partial differential equation for heat conduction needs to be discretised. Equation for heat conduction can be seen in \eqref{equation:Aufgabe1:heatflow} {[}\hyperlink{cite.Aufgabe1:id6}{3}{]}:
\begin{equation}\label{equation:Aufgabe1:heatflow}
\begin{split}\begin{align}
\rho \cdot c \cdot \frac{dT}{dt} = \nabla (\lambda \nabla T) 
\end{align}\end{split}
\end{equation}
\sphinxAtStartPar
reduced to one dimension:
\begin{equation}\label{equation:Aufgabe1:heatflow2}
\begin{split}\begin{align}
\rho \cdot c \cdot \frac{dT}{dt} =  \frac{d}{dx}(\lambda \frac{dT}{dx}) 
\end{align}\end{split}
\end{equation}

\subsection{Space discretication}
\label{\detokenize{Aufgabe1:space-discretication}}
\sphinxAtStartPar
\(R_{ci} =0\)

\sphinxAtStartPar
To calculate the temperaturefield cells and cunduction between them are represented by RC\sphinxhyphen{}Networks. Each cell is represented by a resistance and the conductivity to the neighboring cells. The conductivity between Interrior cells are calculated as follows (For sake of simplicity the surface resistance between layers is neglected. \( R_{ci} =0 \) ) {[}\hyperlink{cite.Aufgabe1:id7}{2}{]}:
\begin{equation}\label{equation:Aufgabe1:conductivityInterior}
\begin{split}\begin{align}
 K_{i-0.5} = \frac{1}{\frac{0.5 \Delta x_{i-1}}{\lambda_{i-1}} + R_{ci} + \frac{0.5 \Delta x_{i}}{\lambda_{i}}} 
\end{align}\end{split}
\end{equation}
\sphinxAtStartPar
For a cell inside a layer (\(\lambda_{i-1} = \lambda_{i} = \lambda_{i+1}\) and \(\Delta x_{i-1} = \Delta x_{i} = \Delta x_{i+1}\)) the conductivity to neighbouring cells collapses to:
\begin{equation}\label{equation:Aufgabe1:conductivityInterior2}
\begin{split}K_{i-0.5} = K_{i+0.5} = \frac{\lambda_{i}}{\Delta x_{i}}\end{split}
\end{equation}
\sphinxAtStartPar
And for Cells at the boundary:
\begin{equation}\label{equation:Aufgabe1:conductivityExterior}
\begin{split}K_{BC,1} = K_{0.5} = \frac{1}{R_{s,1} + \frac{0.5 \Delta x_{1}}{\lambda_{1}}}\end{split}
\end{equation}

\subsection{Time Integration}
\label{\detokenize{Aufgabe1:time-integration}}
\sphinxAtStartPar
using the explicit euler sheme the next Timestep can be calculated as follows:
\begin{equation}\label{equation:Aufgabe1:tempfieldcells}
\begin{split}\begin{align}
    &\text{For cells at the left boundary:}& &T_1^{n+1} = T_1^n+F_{o,1}^* \cdot (K_{BC,1} \cdot   T_{BC}^n  - (K_{BC,l}+K_{2}) \cdot   T_{1}^n + K_{2} \cdot   T_{2}^n)&\\
    &\text{For interior cells:}& &T_i^{n+1} = T_1^n+F_{o,i}^* \cdot (K_{i-1} \cdot   T_{i-1}^n  - (K_{i-1}+K_{i+1}) \cdot   T_{i}^n + K_{i+1} \cdot   T_{i+1}^n)&\\
    &\text{For cells at the right boundary:}& &T_j^{n+1} = T_j^n+F_{o,j}^* \cdot (K_{j-1} \cdot   T_{j-1}^n  - (K_{j-1}+K_{BC,2}) \cdot   T_{j}^n + K_{BC,2} \cdot   T_{BC,2}^n)&\\
\end{align}\end{split}
\end{equation}
\sphinxAtStartPar
or generally written (based on {[}\hyperlink{cite.Aufgabe1:id6}{3}{]}):
\begin{equation}\label{equation:Aufgabe1:tempfield}
\begin{split}T^{n+1} = T^n + F_{o}^* \cdot K \cdot T^n\end{split}
\end{equation}
\sphinxAtStartPar
With the conductivity matrix \( K \):
\begin{equation*}
\begin{split} K = \left [ 
\begin{array}{ccccccc}
    0 & 0 & 0 & 0 & (...) & 0 & 0 & 0  & 0\\
    K_{BC,1} & -(K_{BC,1} + K_w) & K_w & 0 & (...) & 0 & 0 & 0 & 0\\
    0 & K_e & -(K_e + K_w) & K_w & (...) & 0 & 0  & 0 & 0\\
    (...) & (...) & (...) & (...) & (...) & (...) & (...)  & (...) & (...)\\
    0 & 0 & 0 & 0 &(...) & K_{e} & -(K_e + K_w) & K_w & 0 \\
    0 & 0 & 0 & 0 &(...) & 0 & K_{e} & -(K_e + K_{BC,2}) & K_{BC,2} \\
    0 & 0 & 0 & 0 &(...) & 0 & 0 & 0 & 0 \\
\end{array}
\right] \end{split}
\end{equation*}
\sphinxAtStartPar
and the adapted “resistance” matrix \(F_{o}^*\) (layers a,b,c):
\begin{equation*}
\begin{split} F_{o}^* = \left [ 
\begin{array}{ccccccc}
    F_{o,a}^* & 0\\
    0 & F_{o,a}^*  & 0\\ 
     &  0 &  (...)  & 0\\
      &   &  0 & F_{o,b}^* & 0\\
      &   &    & 0 & F_{o,b}^* & 0\\
      &   &   &    & 0  &  (...) & 0\\
      &   &   &   &    & 0 &  F_{o,c}^* & 0\\
\end{array}
\right]  \quad F_{o,i}^* = \frac{\Delta t}{\Delta x_i} \cdot \frac{1}{\rho_i \cdot c_i}\end{split}
\end{equation*}
\sphinxAtStartPar
with the heat capacity \(c_i\) and the density \(\rho_i\).


\subsubsection{Stability}
\label{\detokenize{Aufgabe1:stability}}

\subsubsection{Steady state criteria}
\label{\detokenize{Aufgabe1:steady-state-criteria}}
\sphinxAtStartPar
The simulation stops once a steady state is achieved. The ISO 10211 {[}\hyperlink{cite.Aufgabe1:id8}{1}{]} norm gives a steady state criterea for thermal bridges in building constructions. For iterative solution techniques the following limit is provided:
\begin{equation}\label{equation:Aufgabe1:label}
\begin{split} \sum q_{in} - \frac{\sum q}{2} \le 0.0001 \end{split}
\end{equation}
\sphinxAtStartPar
As derived in the exercise the heatflow through the surface of a finite volume can be approximated as follows:
\begin{equation}\label{equation:Aufgabe1:flows}
\begin{split}\begin{align}
q_{e} = \lambda_e \cdot \frac{T_E-T_P}{\Delta x_e} \\
q_{w} = \lambda_w \cdot \frac{T_P-T_W}{\Delta x_w}
\end{align}\end{split}
\end{equation}
\sphinxAtStartPar
Specifing for the first and last cells with \(\Delta x /2\) and \(R_{se}\) / \(R_{si}\) the heatflow in and out of the construction can be written as:
\begin{equation}\label{equation:Aufgabe1:label}
\begin{split} q_{in} = K_{BC1} \cdot (T_{BC1}-T_{1}) \end{split}
\end{equation}\begin{equation}\label{equation:Aufgabe1:label}
\begin{split} q_{out} = K_{BC2} \cdot (T_{BC2}-T_{n}) \end{split}
\end{equation}

\subsection{Implementation}
\label{\detokenize{Aufgabe1:implementation}}
\begin{sphinxuseclass}{cell}\begin{sphinxVerbatimInput}

\begin{sphinxuseclass}{cell_input}
\begin{sphinxVerbatim}[commandchars=\\\{\}]
\PYG{o}{\PYGZpc{}}\PYG{k}{reset}

\PYG{c+c1}{\PYGZsh{}Libraries}
\PYG{k+kn}{import} \PYG{n+nn}{numpy} \PYG{k}{as} \PYG{n+nn}{np}
\PYG{k+kn}{import} \PYG{n+nn}{matplotlib}\PYG{n+nn}{.}\PYG{n+nn}{pyplot} \PYG{k}{as} \PYG{n+nn}{plt}
\end{sphinxVerbatim}

\end{sphinxuseclass}\end{sphinxVerbatimInput}

\end{sphinxuseclass}
\begin{sphinxuseclass}{cell}\begin{sphinxVerbatimInput}

\begin{sphinxuseclass}{cell_input}
\begin{sphinxVerbatim}[commandchars=\\\{\}]
\PYG{k}{class} \PYG{n+nc}{boundary}\PYG{p}{:}
    \PYG{k}{def} \PYG{n+nf+fm}{\PYGZus{}\PYGZus{}init\PYGZus{}\PYGZus{}}\PYG{p}{(}\PYG{n+nb+bp}{self}\PYG{p}{,} \PYG{n}{temperature1}\PYG{p}{,} \PYG{n}{temperature2}\PYG{p}{,} \PYG{n}{resistance1}\PYG{p}{,} \PYG{n}{resistance2}\PYG{p}{)}\PYG{p}{:}
        \PYG{n+nb+bp}{self}\PYG{o}{.}\PYG{n}{temperature1} \PYG{o}{=} \PYG{n}{temperature1}
        \PYG{n+nb+bp}{self}\PYG{o}{.}\PYG{n}{temperature2} \PYG{o}{=} \PYG{n}{temperature2}
        \PYG{n+nb+bp}{self}\PYG{o}{.}\PYG{n}{resistance1} \PYG{o}{=} \PYG{n}{resistance1}
        \PYG{n+nb+bp}{self}\PYG{o}{.}\PYG{n}{resistance2} \PYG{o}{=} \PYG{n}{resistance2}
\end{sphinxVerbatim}

\end{sphinxuseclass}\end{sphinxVerbatimInput}

\end{sphinxuseclass}
\begin{sphinxuseclass}{cell}\begin{sphinxVerbatimInput}

\begin{sphinxuseclass}{cell_input}
\begin{sphinxVerbatim}[commandchars=\\\{\}]
\PYG{k}{class} \PYG{n+nc}{layer}\PYG{p}{:}
    \PYG{c+c1}{\PYGZsh{} n summarizes the number of all cells}
    \PYG{c+c1}{\PYGZsh{} start value for n = number of boundaries, makes place for boundary temp}
    \PYG{n}{n}\PYG{o}{=}\PYG{l+m+mi}{2}
    \PYG{c+c1}{\PYGZsh{} width sumarizes the thickness of all layers}
    \PYG{n}{width\PYGZus{}sum} \PYG{o}{=}\PYG{l+m+mi}{0}

    \PYG{k}{def} \PYG{n+nf+fm}{\PYGZus{}\PYGZus{}init\PYGZus{}\PYGZus{}}\PYG{p}{(}\PYG{n+nb+bp}{self}\PYG{p}{,} \PYG{n}{material}\PYG{p}{,} \PYG{n}{width}\PYG{p}{,} \PYG{n}{n\PYGZus{}cells}\PYG{p}{,} \PYG{n}{thermConduct}\PYG{p}{,} \PYG{n}{heatCap}\PYG{p}{,} \PYG{n}{density}\PYG{p}{)}\PYG{p}{:}
        \PYG{n+nb+bp}{self}\PYG{o}{.}\PYG{n}{material} \PYG{o}{=} \PYG{n}{material}
        \PYG{n+nb+bp}{self}\PYG{o}{.}\PYG{n}{width} \PYG{o}{=} \PYG{n}{width}
        \PYG{n+nb+bp}{self}\PYG{o}{.}\PYG{n}{n\PYGZus{}cells} \PYG{o}{=} \PYG{n}{n\PYGZus{}cells}
        \PYG{n+nb+bp}{self}\PYG{o}{.}\PYG{n}{thermConduct} \PYG{o}{=} \PYG{n}{thermConduct}
        \PYG{n+nb+bp}{self}\PYG{o}{.}\PYG{n}{heatCap} \PYG{o}{=} \PYG{n}{heatCap}
        \PYG{n+nb+bp}{self}\PYG{o}{.}\PYG{n}{density} \PYG{o}{=} \PYG{n}{density}

        \PYG{c+c1}{\PYGZsh{} Adding number of cells and width to class counters}
        \PYG{n}{layer}\PYG{o}{.}\PYG{n}{n} \PYG{o}{+}\PYG{o}{=} \PYG{n+nb+bp}{self}\PYG{o}{.}\PYG{n}{n\PYGZus{}cells}
        \PYG{n}{layer}\PYG{o}{.}\PYG{n}{width\PYGZus{}sum} \PYG{o}{+}\PYG{o}{=} \PYG{n+nb+bp}{self}\PYG{o}{.}\PYG{n}{width}
\end{sphinxVerbatim}

\end{sphinxuseclass}\end{sphinxVerbatimInput}

\end{sphinxuseclass}

\subsubsection{construction of the layer and boundary objects}
\label{\detokenize{Aufgabe1:construction-of-the-layer-and-boundary-objects}}
\begin{sphinxuseclass}{cell}\begin{sphinxVerbatimInput}

\begin{sphinxuseclass}{cell_input}
\begin{sphinxVerbatim}[commandchars=\\\{\}]
\PYG{n}{boundaries} \PYG{o}{=} \PYG{n}{boundary}\PYG{p}{(}\PYG{l+m+mi}{0}\PYG{p}{,}\PYG{l+m+mi}{20}\PYG{p}{,}\PYG{l+m+mf}{0.04}\PYG{p}{,}\PYG{l+m+mf}{0.10}\PYG{p}{)}

\PYG{c+c1}{\PYGZsh{}Beware, each layer must have at least 3 cells!}
\PYG{n}{layers} \PYG{o}{=} \PYG{p}{[}\PYG{p}{]}
\PYG{n}{layers}\PYG{o}{.}\PYG{n}{append}\PYG{p}{(}\PYG{n}{layer}\PYG{p}{(}\PYG{l+s+s2}{\PYGZdq{}}\PYG{l+s+s2}{Dämmung}\PYG{l+s+s2}{\PYGZdq{}}\PYG{p}{,} \PYG{l+m+mf}{0.2}\PYG{p}{,} \PYG{l+m+mi}{20}\PYG{p}{,} \PYG{l+m+mf}{0.04}\PYG{p}{,} \PYG{l+m+mi}{1470}\PYG{p}{,} \PYG{l+m+mi}{1000}\PYG{p}{)}\PYG{p}{)}
\PYG{n}{layers}\PYG{o}{.}\PYG{n}{append}\PYG{p}{(}\PYG{n}{layer}\PYG{p}{(}\PYG{l+s+s2}{\PYGZdq{}}\PYG{l+s+s2}{STB}\PYG{l+s+s2}{\PYGZdq{}}\PYG{p}{,} \PYG{l+m+mf}{0.15}\PYG{p}{,} \PYG{l+m+mi}{15}\PYG{p}{,} \PYG{l+m+mf}{2.3}\PYG{p}{,} \PYG{l+m+mi}{1000}\PYG{p}{,} \PYG{l+m+mi}{2500}\PYG{p}{)}\PYG{p}{)}
\PYG{n}{layers}\PYG{o}{.}\PYG{n}{append}\PYG{p}{(}\PYG{n}{layer}\PYG{p}{(}\PYG{l+s+s2}{\PYGZdq{}}\PYG{l+s+s2}{Putz}\PYG{l+s+s2}{\PYGZdq{}}\PYG{p}{,} \PYG{l+m+mf}{0.02}\PYG{p}{,} \PYG{l+m+mi}{4}\PYG{p}{,} \PYG{l+m+mf}{0.9}\PYG{p}{,} \PYG{l+m+mi}{1000}\PYG{p}{,} \PYG{l+m+mi}{2000}\PYG{p}{)}\PYG{p}{)}
\end{sphinxVerbatim}

\end{sphinxuseclass}\end{sphinxVerbatimInput}

\end{sphinxuseclass}

\subsubsection{Definition of the simulation time and the step size.}
\label{\detokenize{Aufgabe1:definition-of-the-simulation-time-and-the-step-size}}
\begin{sphinxuseclass}{cell}\begin{sphinxVerbatimInput}

\begin{sphinxuseclass}{cell_input}
\begin{sphinxVerbatim}[commandchars=\\\{\}]
\PYG{c+c1}{\PYGZsh{}Time steps}
\PYG{n}{ts} \PYG{o}{=} \PYG{l+m+mi}{800000}
\PYG{c+c1}{\PYGZsh{}simulation time}
\PYG{n}{tsim}\PYG{o}{=} \PYG{l+m+mi}{3000000} \PYG{c+c1}{\PYGZsh{}s}
\PYG{c+c1}{\PYGZsh{}step sizes}
\PYG{n}{deltat} \PYG{o}{=} \PYG{n}{tsim}\PYG{o}{/}\PYG{n}{ts}
\end{sphinxVerbatim}

\end{sphinxuseclass}\end{sphinxVerbatimInput}

\end{sphinxuseclass}

\subsubsection{Construction of the \protect\(K\protect\) and \protect\(F_o^*\protect\) matrices}
\label{\detokenize{Aufgabe1:construction-of-the-k-and-f-o-matrices}}
\begin{sphinxuseclass}{cell}\begin{sphinxVerbatimInput}

\begin{sphinxuseclass}{cell_input}
\begin{sphinxVerbatim}[commandchars=\\\{\}]
\PYG{c+c1}{\PYGZsh{} reference number of cells to increase readability}
\PYG{n}{n} \PYG{o}{=} \PYG{n}{layer}\PYG{o}{.}\PYG{n}{n}

\PYG{c+c1}{\PYGZsh{} Initialising conductivity matrix K, \PYGZdq{}resistance\PYGZdq{} array Fo, and the error array deltaGrad}
\PYG{n}{K} \PYG{o}{=} \PYG{n}{np}\PYG{o}{.}\PYG{n}{zeros}\PYG{p}{(}\PYG{p}{(}\PYG{n}{n}\PYG{p}{,}\PYG{n}{n}\PYG{p}{)}\PYG{p}{)}
\PYG{n}{Fo} \PYG{o}{=} \PYG{n}{np}\PYG{o}{.}\PYG{n}{zeros}\PYG{p}{(}\PYG{p}{(}\PYG{n}{n}\PYG{p}{,}\PYG{n}{n}\PYG{p}{)}\PYG{p}{)}

\PYG{c+c1}{\PYGZsh{} iterater to keep track of cell numbers}
\PYG{n}{counter} \PYG{o}{=} \PYG{l+m+mi}{1}
\PYG{k}{for} \PYG{n}{layer} \PYG{o+ow}{in} \PYG{n}{layers}\PYG{p}{:}
    \PYG{c+c1}{\PYGZsh{}step size for the current layer}
    \PYG{n}{deltaX}\PYG{o}{=}\PYG{n}{layer}\PYG{o}{.}\PYG{n}{width} \PYG{o}{/} \PYG{n}{layer}\PYG{o}{.}\PYG{n}{n\PYGZus{}cells}

    \PYG{c+c1}{\PYGZsh{}initialize empty conductivity matrix for the current layer}
    \PYG{n}{K\PYGZus{}current} \PYG{o}{=} \PYG{n}{np}\PYG{o}{.}\PYG{n}{zeros}\PYG{p}{(}\PYG{p}{(}\PYG{n}{n}\PYG{p}{,}\PYG{n}{n}\PYG{p}{)}\PYG{p}{)}

    \PYG{c+c1}{\PYGZsh{}defining the \PYGZdq{}east\PYGZdq{} and \PYGZdq{}west\PYGZdq{} conductivity for interior cells}
    \PYG{n}{K\PYGZus{}current}\PYG{p}{[}\PYG{n+nb}{range}\PYG{p}{(}\PYG{n}{counter}\PYG{o}{+}\PYG{l+m+mi}{1}\PYG{p}{,}\PYG{n}{counter}\PYG{o}{+}\PYG{n}{layer}\PYG{o}{.}\PYG{n}{n\PYGZus{}cells}\PYG{p}{)}\PYG{p}{,}\PYG{n+nb}{range}\PYG{p}{(}\PYG{n}{counter}\PYG{p}{,}\PYG{n}{counter}\PYG{o}{+}\PYG{n}{layer}\PYG{o}{.}\PYG{n}{n\PYGZus{}cells}\PYG{o}{\PYGZhy{}}\PYG{l+m+mi}{1}\PYG{p}{)}\PYG{p}{]}\PYG{o}{=} \PYG{n}{layer}\PYG{o}{.}\PYG{n}{thermConduct} \PYG{o}{/} \PYG{n}{deltaX}
    \PYG{n}{K\PYGZus{}current}\PYG{p}{[}\PYG{n+nb}{range}\PYG{p}{(}\PYG{n}{counter}\PYG{p}{,}\PYG{n}{counter}\PYG{o}{+}\PYG{n}{layer}\PYG{o}{.}\PYG{n}{n\PYGZus{}cells}\PYG{o}{\PYGZhy{}}\PYG{l+m+mi}{1}\PYG{p}{)}\PYG{p}{,}\PYG{n+nb}{range}\PYG{p}{(}\PYG{n}{counter}\PYG{o}{+}\PYG{l+m+mi}{1}\PYG{p}{,}\PYG{n}{counter}\PYG{o}{+}\PYG{n}{layer}\PYG{o}{.}\PYG{n}{n\PYGZus{}cells}\PYG{p}{)}\PYG{p}{]}\PYG{o}{=} \PYG{n}{layer}\PYG{o}{.}\PYG{n}{thermConduct} \PYG{o}{/} \PYG{n}{deltaX}
    
    \PYG{c+c1}{\PYGZsh{}defining the \PYGZdq{}east\PYGZdq{} and \PYGZdq{}west\PYGZdq{} conductivity for the last and first cell of each layer}
    \PYG{c+c1}{\PYGZsh{}these are added up between layers, with half the cell distance}
    \PYG{n}{K\PYGZus{}current}\PYG{p}{[}\PYG{p}{(}\PYG{n}{counter}\PYG{p}{,}\PYG{n}{counter}\PYG{o}{+}\PYG{n}{layer}\PYG{o}{.}\PYG{n}{n\PYGZus{}cells}\PYG{p}{)}\PYG{p}{,}\PYG{p}{(}\PYG{n}{counter}\PYG{o}{\PYGZhy{}}\PYG{l+m+mi}{1}\PYG{p}{,}\PYG{n}{counter}\PYG{o}{+}\PYG{n}{layer}\PYG{o}{.}\PYG{n}{n\PYGZus{}cells}\PYG{o}{\PYGZhy{}}\PYG{l+m+mi}{1}\PYG{p}{)}\PYG{p}{]}\PYG{o}{=} \PYG{n}{deltaX} \PYG{o}{/} \PYG{p}{(}\PYG{l+m+mi}{2} \PYG{o}{*} \PYG{n}{layer}\PYG{o}{.}\PYG{n}{thermConduct}\PYG{p}{)}   
    \PYG{n}{K\PYGZus{}current}\PYG{p}{[}\PYG{p}{(}\PYG{n}{counter}\PYG{o}{\PYGZhy{}}\PYG{l+m+mi}{1}\PYG{p}{,}\PYG{n}{counter}\PYG{o}{+}\PYG{n}{layer}\PYG{o}{.}\PYG{n}{n\PYGZus{}cells}\PYG{o}{\PYGZhy{}}\PYG{l+m+mi}{1}\PYG{p}{)}\PYG{p}{,}\PYG{p}{(}\PYG{n}{counter}\PYG{p}{,}\PYG{n}{counter}\PYG{o}{+}\PYG{n}{layer}\PYG{o}{.}\PYG{n}{n\PYGZus{}cells}\PYG{p}{)}\PYG{p}{]}\PYG{o}{=} \PYG{n}{deltaX} \PYG{o}{/} \PYG{p}{(}\PYG{l+m+mi}{2} \PYG{o}{*} \PYG{n}{layer}\PYG{o}{.}\PYG{n}{thermConduct}\PYG{p}{)}
    
    \PYG{c+c1}{\PYGZsh{} Filling the \PYGZdq{}resistance\PYGZdq{} matrix for each layer}
    \PYG{n}{Fo}\PYG{p}{[}\PYG{n+nb}{range}\PYG{p}{(}\PYG{n}{counter}\PYG{p}{,}\PYG{n}{counter}\PYG{o}{+}\PYG{n}{layer}\PYG{o}{.}\PYG{n}{n\PYGZus{}cells}\PYG{p}{)}\PYG{p}{,}\PYG{n+nb}{range}\PYG{p}{(}\PYG{n}{counter}\PYG{p}{,}\PYG{n}{counter}\PYG{o}{+}\PYG{n}{layer}\PYG{o}{.}\PYG{n}{n\PYGZus{}cells}\PYG{p}{)}\PYG{p}{]} \PYG{o}{=} \PYG{n}{deltat} \PYG{o}{/}\PYG{p}{(}\PYG{n}{layer}\PYG{o}{.}\PYG{n}{density} \PYG{o}{*} \PYG{n}{layer}\PYG{o}{.}\PYG{n}{heatCap} \PYG{o}{*} \PYG{n}{deltaX}\PYG{p}{)}

    \PYG{n}{K} \PYG{o}{+}\PYG{o}{=} \PYG{n}{K\PYGZus{}current}
    
    \PYG{c+c1}{\PYGZsh{}Inverting the Konductivity between surfaces}
    \PYG{n}{K}\PYG{p}{[}\PYG{p}{(}\PYG{n}{counter}\PYG{p}{)}\PYG{p}{,}\PYG{p}{(}\PYG{n}{counter}\PYG{o}{\PYGZhy{}}\PYG{l+m+mi}{1}\PYG{p}{)}\PYG{p}{]} \PYG{o}{=} \PYG{l+m+mi}{1}\PYG{o}{/}\PYG{n}{K}\PYG{p}{[}\PYG{p}{(}\PYG{n}{counter}\PYG{p}{)}\PYG{p}{,}\PYG{p}{(}\PYG{n}{counter}\PYG{o}{\PYGZhy{}}\PYG{l+m+mi}{1}\PYG{p}{)}\PYG{p}{]}
    \PYG{n}{K}\PYG{p}{[}\PYG{p}{(}\PYG{n}{counter}\PYG{o}{\PYGZhy{}}\PYG{l+m+mi}{1}\PYG{p}{)}\PYG{p}{,}\PYG{p}{(}\PYG{n}{counter}\PYG{p}{)}\PYG{p}{]} \PYG{o}{=} \PYG{l+m+mi}{1}\PYG{o}{/}\PYG{n}{K}\PYG{p}{[}\PYG{p}{(}\PYG{n}{counter}\PYG{o}{\PYGZhy{}}\PYG{l+m+mi}{1}\PYG{p}{)}\PYG{p}{,}\PYG{p}{(}\PYG{n}{counter}\PYG{p}{)}\PYG{p}{]}

    \PYG{n}{counter} \PYG{o}{+}\PYG{o}{=} \PYG{n}{layer}\PYG{o}{.}\PYG{n}{n\PYGZus{}cells}
\end{sphinxVerbatim}

\end{sphinxuseclass}\end{sphinxVerbatimInput}

\end{sphinxuseclass}
\begin{sphinxuseclass}{cell}\begin{sphinxVerbatimInput}

\begin{sphinxuseclass}{cell_input}
\begin{sphinxVerbatim}[commandchars=\\\{\}]
\PYG{c+c1}{\PYGZsh{} Boundary Conditions}
\PYG{n}{K\PYGZus{}l} \PYG{o}{=} \PYG{l+m+mi}{1} \PYG{o}{/} \PYG{p}{(}\PYG{n}{boundaries}\PYG{o}{.}\PYG{n}{resistance1} \PYG{o}{+} \PYG{l+m+mf}{0.5}\PYG{o}{*} \PYG{p}{(}\PYG{n}{layers}\PYG{p}{[}\PYG{l+m+mi}{0}\PYG{p}{]}\PYG{o}{.}\PYG{n}{width} \PYG{o}{/} \PYG{n}{layers}\PYG{p}{[}\PYG{l+m+mi}{0}\PYG{p}{]}\PYG{o}{.}\PYG{n}{n\PYGZus{}cells}\PYG{p}{)} \PYG{o}{/} \PYG{n}{layers}\PYG{p}{[}\PYG{l+m+mi}{0}\PYG{p}{]}\PYG{o}{.}\PYG{n}{thermConduct}\PYG{p}{)}
\PYG{n}{K\PYGZus{}r} \PYG{o}{=} \PYG{l+m+mi}{1} \PYG{o}{/} \PYG{p}{(}\PYG{n}{boundaries}\PYG{o}{.}\PYG{n}{resistance2} \PYG{o}{+} \PYG{l+m+mf}{0.5}\PYG{o}{*} \PYG{p}{(}\PYG{n}{layers}\PYG{p}{[}\PYG{o}{\PYGZhy{}}\PYG{l+m+mi}{1}\PYG{p}{]}\PYG{o}{.}\PYG{n}{width} \PYG{o}{/} \PYG{n}{layers}\PYG{p}{[}\PYG{o}{\PYGZhy{}}\PYG{l+m+mi}{1}\PYG{p}{]}\PYG{o}{.}\PYG{n}{n\PYGZus{}cells}\PYG{p}{)} \PYG{o}{/} \PYG{n}{layers}\PYG{p}{[}\PYG{o}{\PYGZhy{}}\PYG{l+m+mi}{1}\PYG{p}{]}\PYG{o}{.}\PYG{n}{thermConduct}\PYG{p}{)}

\PYG{n}{K}\PYG{p}{[}\PYG{l+m+mi}{1}\PYG{p}{,}\PYG{l+m+mi}{0}\PYG{p}{]} \PYG{o}{=} \PYG{n}{K\PYGZus{}l}
\PYG{n}{K}\PYG{p}{[}\PYG{n+nb}{len}\PYG{p}{(}\PYG{n}{K}\PYG{p}{)}\PYG{o}{\PYGZhy{}}\PYG{l+m+mi}{2}\PYG{p}{,}\PYG{n+nb}{len}\PYG{p}{(}\PYG{n}{K}\PYG{p}{)}\PYG{o}{\PYGZhy{}}\PYG{l+m+mi}{1}\PYG{p}{]} \PYG{o}{=} \PYG{n}{K\PYGZus{}r}
\end{sphinxVerbatim}

\end{sphinxuseclass}\end{sphinxVerbatimInput}

\end{sphinxuseclass}
\begin{sphinxuseclass}{cell}\begin{sphinxVerbatimInput}

\begin{sphinxuseclass}{cell_input}
\begin{sphinxVerbatim}[commandchars=\\\{\}]
\PYG{c+c1}{\PYGZsh{}The diagonal of the conductivity matrix is defined as the negative sum of the }
\PYG{c+c1}{\PYGZsh{}\PYGZdq{}east\PYGZdq{} and \PYGZdq{}west\PYGZdq{} conductivity for the cell}
\PYG{n}{K}\PYG{p}{[}\PYG{n+nb}{range}\PYG{p}{(}\PYG{l+m+mi}{0}\PYG{p}{,}\PYG{n+nb}{len}\PYG{p}{(}\PYG{n}{K}\PYG{p}{)}\PYG{p}{)}\PYG{p}{,}\PYG{n+nb}{range}\PYG{p}{(}\PYG{l+m+mi}{0}\PYG{p}{,}\PYG{n+nb}{len}\PYG{p}{(}\PYG{n}{K}\PYG{p}{)}\PYG{p}{)}\PYG{p}{]}\PYG{o}{=} \PYG{o}{\PYGZhy{}}\PYG{n}{np}\PYG{o}{.}\PYG{n}{sum}\PYG{p}{(}\PYG{n}{K}\PYG{p}{,}\PYG{n}{axis}\PYG{o}{=}\PYG{l+m+mi}{1}\PYG{p}{)}

\PYG{c+c1}{\PYGZsh{}free first and last row; this leads to constant temperatures at the boundaries}
\PYG{n}{K}\PYG{p}{[}\PYG{l+m+mi}{0}\PYG{p}{,}\PYG{p}{:}\PYG{p}{]}\PYG{o}{=}\PYG{l+m+mi}{0}
\PYG{n}{K}\PYG{p}{[}\PYG{o}{\PYGZhy{}}\PYG{l+m+mi}{1}\PYG{p}{,}\PYG{p}{:}\PYG{p}{]}\PYG{o}{=}\PYG{l+m+mi}{0}
\end{sphinxVerbatim}

\end{sphinxuseclass}\end{sphinxVerbatimInput}

\end{sphinxuseclass}
\begin{sphinxuseclass}{cell}\begin{sphinxVerbatimInput}

\begin{sphinxuseclass}{cell_input}
\begin{sphinxVerbatim}[commandchars=\\\{\}]
\PYG{c+c1}{\PYGZsh{}initialize the temperaturefield}
\PYG{n}{T} \PYG{o}{=} \PYG{n}{np}\PYG{o}{.}\PYG{n}{ones}\PYG{p}{(}\PYG{n}{layer}\PYG{o}{.}\PYG{n}{n}\PYG{p}{)} \PYG{o}{*} \PYG{l+m+mi}{10}

\PYG{c+c1}{\PYGZsh{}Insert Boundary Konditions}
\PYG{n}{T}\PYG{p}{[}\PYG{l+m+mi}{0}\PYG{p}{]}\PYG{o}{=} \PYG{n}{boundaries}\PYG{o}{.}\PYG{n}{temperature1}
\PYG{n}{T}\PYG{p}{[}\PYG{o}{\PYGZhy{}}\PYG{l+m+mi}{1}\PYG{p}{]}\PYG{o}{=} \PYG{n}{boundaries}\PYG{o}{.}\PYG{n}{temperature2}
\end{sphinxVerbatim}

\end{sphinxuseclass}\end{sphinxVerbatimInput}

\end{sphinxuseclass}
\begin{sphinxuseclass}{cell}\begin{sphinxVerbatimInput}

\begin{sphinxuseclass}{cell_input}
\begin{sphinxVerbatim}[commandchars=\\\{\}]
\PYG{c+c1}{\PYGZsh{}Calculate the adjusted Matrix outside the loop as both are time independent}
\PYG{n}{K}\PYG{o}{=}\PYG{n}{np}\PYG{o}{.}\PYG{n}{dot}\PYG{p}{(}\PYG{n}{Fo}\PYG{p}{,}\PYG{n}{K}\PYG{p}{)}
\PYG{n}{T\PYGZus{}plus} \PYG{o}{=}  \PYG{n}{np}\PYG{o}{.}\PYG{n}{dot}\PYG{p}{(}\PYG{n}{K}\PYG{p}{,}\PYG{n}{T}\PYG{p}{)} \PYG{o}{+} \PYG{n}{T}

\PYG{c+c1}{\PYGZsh{}Initialize heatflow for the loop criteria}
\PYG{n}{qin} \PYG{o}{=} \PYG{l+m+mi}{1}
\PYG{n}{qout} \PYG{o}{=} \PYG{l+m+mi}{0}

\PYG{n}{t}\PYG{o}{=}\PYG{l+m+mi}{0}
\PYG{k}{while} \PYG{n}{qin} \PYG{o}{\PYGZhy{}} \PYG{p}{(}\PYG{n}{qin}\PYG{o}{+}\PYG{n}{qout}\PYG{p}{)}\PYG{o}{/}\PYG{l+m+mi}{2} \PYG{o}{\PYGZgt{}} \PYG{l+m+mf}{0.0001}\PYG{p}{:}
    \PYG{n}{T}\PYG{o}{=}\PYG{n}{T\PYGZus{}plus}
    \PYG{n}{T\PYGZus{}plus} \PYG{o}{=}  \PYG{n}{np}\PYG{o}{.}\PYG{n}{dot}\PYG{p}{(}\PYG{n}{K}\PYG{p}{,}\PYG{n}{T}\PYG{p}{)} \PYG{o}{+} \PYG{n}{T}
    
    \PYG{c+c1}{\PYGZsh{}Calculate the heatflow for the break criteria}
    \PYG{n}{qout} \PYG{o}{=} \PYG{n+nb}{abs}\PYG{p}{(}\PYG{n}{K\PYGZus{}l}\PYG{o}{*}\PYG{p}{(}\PYG{n}{T\PYGZus{}plus}\PYG{p}{[}\PYG{l+m+mi}{1}\PYG{p}{]}\PYG{o}{\PYGZhy{}}\PYG{n}{T\PYGZus{}plus}\PYG{p}{[}\PYG{l+m+mi}{0}\PYG{p}{]}\PYG{p}{)}\PYG{p}{)}
    \PYG{n}{qin} \PYG{o}{=} \PYG{n+nb}{abs}\PYG{p}{(}\PYG{n}{K\PYGZus{}r}\PYG{o}{*}\PYG{p}{(}\PYG{n}{T\PYGZus{}plus}\PYG{p}{[}\PYG{o}{\PYGZhy{}}\PYG{l+m+mi}{1}\PYG{p}{]}\PYG{o}{\PYGZhy{}}\PYG{n}{T\PYGZus{}plus}\PYG{p}{[}\PYG{o}{\PYGZhy{}}\PYG{l+m+mi}{2}\PYG{p}{]}\PYG{p}{)}\PYG{p}{)}
    
    \PYG{n}{t} \PYG{o}{+}\PYG{o}{=} \PYG{n}{deltat}
\end{sphinxVerbatim}

\end{sphinxuseclass}\end{sphinxVerbatimInput}

\end{sphinxuseclass}
\begin{sphinxuseclass}{cell}\begin{sphinxVerbatimInput}

\begin{sphinxuseclass}{cell_input}
\begin{sphinxVerbatim}[commandchars=\\\{\}]
\PYG{n}{x\PYGZus{}pos}\PYG{o}{=} \PYG{n}{np}\PYG{o}{.}\PYG{n}{array}\PYG{p}{(}\PYG{p}{[}\PYG{l+m+mi}{0}\PYG{p}{]}\PYG{p}{)}
\PYG{n}{counter}\PYG{o}{=}\PYG{l+m+mi}{0}
\PYG{n}{width\PYGZus{}sum}\PYG{o}{=}\PYG{l+m+mi}{0}
\PYG{k}{for} \PYG{n}{layer} \PYG{o+ow}{in} \PYG{n}{layers}\PYG{p}{:}
    \PYG{n}{width\PYGZus{}cell} \PYG{o}{=} \PYG{n}{layer}\PYG{o}{.}\PYG{n}{width} \PYG{o}{/} \PYG{n}{layer}\PYG{o}{.}\PYG{n}{n\PYGZus{}cells}
    \PYG{n}{x\PYGZus{}pos} \PYG{o}{=} \PYG{n}{np}\PYG{o}{.}\PYG{n}{append}\PYG{p}{(}\PYG{n}{x\PYGZus{}pos} \PYG{p}{,} \PYG{n}{np}\PYG{o}{.}\PYG{n}{arange}\PYG{p}{(}\PYG{n}{counter}\PYG{o}{+} \PYG{n}{width\PYGZus{}cell}\PYG{o}{/}\PYG{l+m+mi}{2}\PYG{p}{,}\PYG{n}{layer}\PYG{o}{.}\PYG{n}{width}\PYG{o}{+}\PYG{n}{counter}\PYG{p}{,} \PYG{n}{width\PYGZus{}cell}\PYG{p}{)}\PYG{p}{)}
    \PYG{n}{counter} \PYG{o}{+}\PYG{o}{=} \PYG{n}{layer}\PYG{o}{.}\PYG{n}{width}
    \PYG{n}{width\PYGZus{}sum} \PYG{o}{+}\PYG{o}{=} \PYG{n}{layer}\PYG{o}{.}\PYG{n}{width}
    
\PYG{n}{x\PYGZus{}pos} \PYG{o}{=} \PYG{n}{np}\PYG{o}{.}\PYG{n}{append}\PYG{p}{(}\PYG{n}{x\PYGZus{}pos} \PYG{p}{,}\PYG{n}{np}\PYG{o}{.}\PYG{n}{array}\PYG{p}{(}\PYG{p}{[}\PYG{n}{width\PYGZus{}sum}\PYG{p}{]}\PYG{p}{)}\PYG{p}{)}
\end{sphinxVerbatim}

\end{sphinxuseclass}\end{sphinxVerbatimInput}

\end{sphinxuseclass}
\begin{sphinxuseclass}{cell}\begin{sphinxVerbatimInput}

\begin{sphinxuseclass}{cell_input}
\begin{sphinxVerbatim}[commandchars=\\\{\}]
\PYG{n}{plt}\PYG{o}{.}\PYG{n}{xlabel}\PYG{p}{(}\PYG{l+s+s2}{\PYGZdq{}}\PYG{l+s+s2}{x position [m]}\PYG{l+s+s2}{\PYGZdq{}}\PYG{p}{)}
\PYG{n}{plt}\PYG{o}{.}\PYG{n}{ylabel}\PYG{p}{(}\PYG{l+s+s2}{\PYGZdq{}}\PYG{l+s+s2}{Temperature [°C]}\PYG{l+s+s2}{\PYGZdq{}}\PYG{p}{)}
\PYG{n}{plt}\PYG{o}{.}\PYG{n}{plot}\PYG{p}{(}\PYG{n}{x\PYGZus{}pos}\PYG{p}{,} \PYG{n}{T\PYGZus{}plus}\PYG{p}{,} \PYG{l+s+s1}{\PYGZsq{}}\PYG{l+s+s1}{o\PYGZhy{}}\PYG{l+s+s1}{\PYGZsq{}}\PYG{p}{,} \PYG{n}{alpha}\PYG{o}{=}\PYG{l+m+mf}{0.65}\PYG{p}{)}
\PYG{n}{x\PYGZus{}w}\PYG{o}{=}\PYG{l+m+mi}{0}
\PYG{k}{for} \PYG{n}{layer} \PYG{o+ow}{in} \PYG{n}{layers}\PYG{p}{:}
    \PYG{n}{plt}\PYG{o}{.}\PYG{n}{plot}\PYG{p}{(}\PYG{p}{[}\PYG{n}{x\PYGZus{}w}\PYG{p}{,}\PYG{n}{x\PYGZus{}w}\PYG{p}{]}\PYG{p}{,} \PYG{p}{[}\PYG{o}{\PYGZhy{}}\PYG{l+m+mi}{5}\PYG{p}{,}\PYG{l+m+mi}{35}\PYG{p}{]}\PYG{p}{,} \PYG{n}{color}\PYG{o}{=}\PYG{l+s+s1}{\PYGZsq{}}\PYG{l+s+s1}{k}\PYG{l+s+s1}{\PYGZsq{}}\PYG{p}{,} \PYG{n}{alpha}\PYG{o}{=}\PYG{l+m+mf}{0.7}\PYG{p}{)}
    \PYG{n}{x\PYGZus{}w} \PYG{o}{+}\PYG{o}{=} \PYG{n}{layer}\PYG{o}{.}\PYG{n}{width}
\PYG{n}{plt}\PYG{o}{.}\PYG{n}{plot}\PYG{p}{(}\PYG{p}{[}\PYG{n}{layer}\PYG{o}{.}\PYG{n}{width\PYGZus{}sum}\PYG{p}{,}\PYG{n}{layer}\PYG{o}{.}\PYG{n}{width\PYGZus{}sum}\PYG{p}{]}\PYG{p}{,} \PYG{p}{[}\PYG{o}{\PYGZhy{}}\PYG{l+m+mi}{5}\PYG{p}{,}\PYG{l+m+mi}{35}\PYG{p}{]}\PYG{p}{,} \PYG{n}{color}\PYG{o}{=}\PYG{l+s+s1}{\PYGZsq{}}\PYG{l+s+s1}{k}\PYG{l+s+s1}{\PYGZsq{}}\PYG{p}{,} \PYG{n}{alpha}\PYG{o}{=}\PYG{l+m+mf}{0.7}\PYG{p}{)}
\PYG{n}{plt}\PYG{o}{.}\PYG{n}{show}\PYG{p}{(}\PYG{p}{)}
\PYG{n}{a} \PYG{o}{=} \PYG{n}{np}\PYG{o}{.}\PYG{n}{array}\PYG{p}{(}\PYG{p}{[}\PYG{l+m+mi}{1}\PYG{p}{,}\PYG{o}{\PYGZhy{}}\PYG{l+m+mi}{1}\PYG{p}{,}\PYG{l+m+mi}{3}\PYG{p}{]}\PYG{p}{)}
\end{sphinxVerbatim}

\end{sphinxuseclass}\end{sphinxVerbatimInput}
\begin{sphinxVerbatimOutput}

\begin{sphinxuseclass}{cell_output}
\noindent\sphinxincludegraphics{{30caab2b41eff7ed7614f4fe6f861d8191b980402cab11da40f7fb66b14b88e6}.png}

\end{sphinxuseclass}\end{sphinxVerbatimOutput}

\end{sphinxuseclass}

\subsubsection{Determine the time in hours when a 3\sphinxhyphen{}layered component reaches the steady state}
\label{\detokenize{Aufgabe1:determine-the-time-in-hours-when-a-3-layered-component-reaches-the-steady-state}}
\begin{sphinxuseclass}{cell}\begin{sphinxVerbatimInput}

\begin{sphinxuseclass}{cell_input}
\begin{sphinxVerbatim}[commandchars=\\\{\}]
\PYG{n+nb}{print}\PYG{p}{(}\PYG{l+s+s2}{\PYGZdq{}}\PYG{l+s+s2}{time until steady state: }\PYG{l+s+s2}{\PYGZdq{}} \PYG{o}{+} \PYG{n+nb}{str}\PYG{p}{(}\PYG{n+nb}{round}\PYG{p}{(}\PYG{n}{t}\PYG{o}{/}\PYG{l+m+mi}{60}\PYG{o}{/}\PYG{l+m+mi}{60}\PYG{p}{,}\PYG{l+m+mi}{1}\PYG{p}{)}\PYG{p}{)} \PYG{o}{+} \PYG{l+s+s2}{\PYGZdq{}}\PYG{l+s+s2}{ h}\PYG{l+s+s2}{\PYGZdq{}}\PYG{p}{)}
\end{sphinxVerbatim}

\end{sphinxuseclass}\end{sphinxVerbatimInput}
\begin{sphinxVerbatimOutput}

\begin{sphinxuseclass}{cell_output}
\begin{sphinxVerbatim}[commandchars=\\\{\}]
time until steady state: 473.7 h
\end{sphinxVerbatim}

\end{sphinxuseclass}\end{sphinxVerbatimOutput}

\end{sphinxuseclass}

\subsection{References}
\label{\detokenize{Aufgabe1:references}}
\begin{sphinxthebibliography}{1}
\bibitem[1]{Aufgabe1:id8}
\sphinxAtStartPar
ISO/TC 59/SC 13. Iso 10211:2017 thermal bridges in building construction — heat flows and surface temperatures — detailed calculations. International Organization for Standardization, 03 2016.
\bibitem[2]{Aufgabe1:id7}
\sphinxAtStartPar
Carl\sphinxhyphen{}Eric Hagentoft. \sphinxstyleemphasis{Introduction to building physics}. Studentlitteratur, Lund, 2001. URL: \sphinxurl{https://buildingphysicshagentoft.com/text-books/introduction-to-building-physics/}.
\bibitem[3]{Aufgabe1:id6}
\sphinxAtStartPar
E. Walther. \sphinxstyleemphasis{Building Physics Applications in Python}. DIY Spring, Paris, 2021.
\end{sphinxthebibliography}







\renewcommand{\indexname}{Index}
\printindex
\end{document}